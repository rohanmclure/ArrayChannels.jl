\label{sec:arraychannels}

The \texttt{ArrayChannels.jl} library provides synchronous, in-place
communication options for message passing, utilising rendezvous message
passing. This is achieved by providing an \texttt{ArrayChannel}
construct encapsulating a template for the forms of array communication
that must occur. Constructing an \texttt{ArrayChannel} will require
specification of the processes that must participate in relevant
communication events. Each participating process will allocate a buffer
of the same fixed size that will be reused for all communication tasks.

In distributed computation tasks that depend on data to be received as
messages from other processes, reuse of the same message buffers
improves temporal locality. This in turn causes the receiving new
messages and acting upon message contents to be more likely to occur in
cache and with fewer cache misses.

\subsection{Point-to-Point}
\label{sec:p2p}

For point-to-point communication, we provide two commands, \texttt{put!}
and \texttt{take!} which allow processes to send messages to or receive
from a specified process. In code snippet~\ref{code:put-take}, process
one creates an \texttt{ArrayChannel} to facilitate communication between
worker processes. By using a remote call, the programmer may initiate
communication by passing the \texttt{ArrayChannel} as parameters to the
remotecall. \texttt{ArrayChannel} constructs associate with different
data buffers depending on which process interacts with the reference.

\begin{figure}[htb]
  \begin{lstlisting}[language=Julia]
# 10 x 10 buffer
AC = ArrayChannel(Float64, workers(), 10, 10)

@sync begin
  @spawnat 2 begin
    fill!(AC, 1.0)
    put!(AC, 3)
  end
  @spawnat 3 begin
    take!(AC, 2)
    @assert AC[1,1] = 1.0
  end
end
  \end{lstlisting}
  \caption{Message of 10 x 10 matrix of ones sent via point-to-point messaging}
  \label{code:put-take}
\end{figure}

\texttt{put!} and \texttt{take!} operations will block until all buffer
contents have been communicated and written at the recipient's buffer.
\texttt{ArrayChannels.jl} will then
simply take the contents of the input buffer and deposit them in the
output buffer at the recipient process, using the same buffer for each
successive communication operation for improved temporal locality. 

\subsection{Reduction}
\label{sec:reduce}

All participants in the underlying \texttt{ArrayChannel} must signal
their intent on initiating a reduction by calling \texttt{reduce!} on
the remote channel reference, supplying the reduction operator and
`root' process which will receive the resultant data. In figure~\ref{code:reduce}, the master process initialises an
\texttt{ArrayChannel} for the worker processes, and then causes each
participating process to signal for a sum reduction on their local data,
directing the result towards process two. After this reduction has taken
place, only process two's data will be modified. \texttt{reduce!} will
block so long as the calling process is still required to facilitate the
reduction under the current topology.

\begin{figure}[htb]
  \begin{lstlisting}[language=Julia]
AC = ArrayChannel(Int64, workers(), 10)
@sync for proc in workers()
  @spawnat proc begin
    fill!(AC, 1)
    reduce!(+, AC, 2)
  end
end
@assert @fetchfrom 2 AC[1,1] == nworkers()
  \end{lstlisting}
  \caption{Sum reduce of vectors residing on five worker processes}
  \label{code:reduce}
\end{figure}

We implement the tree topology for \texttt{reduce!}, targetting
hierarchical network topologies for distributed clusters. The
\texttt{reduce!} function is defined itself in terms of point-to-point
communication, where processes determine their position in the reduction
topology depending on their process identifier. Since the method is only
intended to alter data residing at the root process, we retain two
buffers in addition to the main \texttt{ArrayChannel} buffer for use in
reduction operations in storing intermediate results.

\subsection{Scatter / Gather}
\label{sec:scatter-gather}

The scatter / gather pattern differs from point-to-point communication and reduction in that all processes, not just those participating in the \texttt{ArrayChannel} may receive and send messages through this mode. Within a `scatter!` operation, a master node will allocate disjoint regions of its local data for communication with some specified worker processes. Every invocation of `scatter!` will block the caller until the processes have completed receiving the data. Conversely, a `gather!` operation causes the master process to wait for workers to send back regions of array data that match the specified directions.

\begin{figure}[htb]
	\begin{lstlisting}[language=Julia]
		
	\end{lstlisting}
	\caption{Both scatter and gather patterns at work.}
	\label{code:scatter-gather
\end{figure}

In code snippet \ref{code:scatter-gather}
